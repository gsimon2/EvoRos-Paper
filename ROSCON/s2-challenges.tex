\paragraph{Challenges.}
Our basic approach is to configure a GA (or other evolutionary algorithm) as a front-end, handling the details of genomes, populations, and selection, with ROS-based simulations invoked to evaluate individual robots (or possibly groups of collaborating robots).  
%
Although this task might appear straightforward, we have encountered some key challenges.

%\textit{Simulation Efficiency.}
First, individual simulations in ROS/Gazebo often require substantially more wall-clock time than those of typical ER  simulators.  
%
With the addition of plugins such as MAVProxy in the Ardupilot stack, this constraint can further slow simulations, minimizing the gain of using simulation versus physical robots.
%
Considering the distributed nature of ROS, and the overhead of multiple plugins/nodes running concurrently, this performance is not surprising.  
%
Fortunately, evaluations are ``pleasantly parallel'' and can often be 
distributed across computing resources to speed up the execution time for each generation.  
%
Depending on the specific plugin stack required, one ROS instance can spawn 
multiple Gazebo instances, or single ROS/Gazebo pairs can be distributed across multiple virtual machines~(VMs).  
%
We have tested both approaches and have sample implementations available~(see the end of this document for Github links).

Another issue in ROS/Gazebo, which we have only partially addressed, is determinism.  
%
Given an identical random seed, most ER researchers would expect to be able to reproduce simulation results exactly.
%
Admittedly, ROS was not necessarily intended for this type of application, and after all the real world is not deterministic.
%
So far, we have been able to reduce the disparity among results of identical evaluations by employing smaller time steps or throttling the update rate of the ROS clock.  
%
We plan to continue investigating this issue moving forward.
