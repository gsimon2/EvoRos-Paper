\vspace{0.2in}
\begin{center}
{\bf \large Description}
\end{center}

\paragraph{Motivations.} 
Evolutionary robotics (ER)~\cite{Floreano2008} applies the basic principles of genetic evolution to the design of robots through the application of the genetic algorithm~(GA).
%
An artificial genome specifies the robot's control system and possibly aspects of its morphology~(body).
%
Individuals in a population are evaluated~(typically in simulation) with respect to one or more tasks, with the best performing individuals selected to pass their genes to the next generation.
%
Evolutionary approaches have yielded effective controllers and physical designs for a variety of crawling, swimming, and flying robots~\cite{bongard-lipson, Lipson2000}.  
%
Our own research has applied evolutionary algorithms to optimize both morphology and control
in aquatic and terrestrial robots~\cite{Clark.JournalBB.2015,MooreALIFEJournal}.  
%
Evolving robot behavior and morphology is interesting in its own right, but from an
engineering perspective, a major advantage of evolutionary search is the possible discovery of solutions (as well as potential problems) that the engineer might not otherwise have considered.

Simulation is an essential component of evolutionary robotics, greatly reducing the time to evolve solutions while avoiding possible damage to physical robots.
%
The ER community typically creates one-off simulation environments from a few different physics 
engines~(e.g., ODE, Bullet, VoxCAD, Simulink) to conduct an experiment.  
%
Environments are sparse, generally featuring the robot and possibly a few obstacles.  
%
Tasks typically comprise locomotion, navigation, and basic problem solving.
%
Robots themselves contain only a few sensors, most often developed for the specific 
experiment being conducted.  
%
Hence, ER tasks are often limited by the scope of the simulation environment and how much time a developer has to code obstacles, sensors, and the platform itself.  
%
Models are not necessarily shareable between developers due to a lack of standardization.  
%
While many research questions can, and have, been answered by simple simulations, 
it becomes difficult to address more complex questions in these environments.  

In contrast, the broader robotics community can address very complex tasks; the robotic systems utilize many sensor modalities to build a coherent understanding of their environment.
%
By providing tested models of commercially available hardware, ROS/Gazebo provides a platform to study high-level behaviors while saving developer time during the design phase.  
%
Additionally, results have been shown to transfer to real robots, potentially addressing the ``reality-gap'' often encountered in ER~\cite{Koos2010}.  

In the {\project} project, we have developed an evolutionary framework that integrates ROS-based simulations for robot evaluation.  
%
Our current prototype includes ROS, Gazebo, Ardupilot and MAVROS.    
%
The primary goals of the project are twofold. 
%
First, the framework enables researchers in ER to take advantage ROS and related simulation tools.
%
Second, it enables robot developers to employ evolutionary search during the design process.

%\pkm{Further discuss/expand on advantages: very active community, lots of new
%features/capabilities being developed, works with Gazebo (and other physics-based simulators)
%providing support for many types of sensors/actuators, code can be directly ported
%to physical robot, Reality gap is a huge issue in ER.}
%
%\pkm{Mention somewhere ``as evidenced by the enormous success of ROSCon and
%large developer community.'' (?)}
%
%\pkm{better wording?}
%\pkm{Briefly describe current status, esp. that we have a version that integrates ROS/Gazebo
%as well as Ardupilot, MAVProxy and Erle-Rover, as well as different ROS/Gazebo/VM configurations.}
%\pkm{Probably mention Github repositories here as well, preliminary with more to come.}

