\begin{abstract}
In this paper, we describe the Evo-ROS framework, which is intended to help bridge the gap between the evolutionary and traditional robotics communities.  
%
Evo-ROS combines an evolutionary algorithm with individual physics-based evaluations conducted using the Robot Operating System (ROS) and the Gazebo simulation environment.
% 
% \ajc{But doesn't Evo-ROS not include the external evolutionary algorithm?}
%
% In this manner, Evo-ROS can integrate and apply evolutionary search methods
% to a wide variety of platforms and sensor/actuator suites developed by the ROS community.  
%
Our goals in developing Evo-ROS are to (1) provide researchers in evolutionary robotics with access to the extensive support for real-world components and capabilities developed by the ROS community and~(2) 
enable ROS developers, and more broadly robotics researchers, to take advantage of evolutionary search during design and testing. 
%
We describe the details of the Evo-ROS structure and operation, followed by
presentation of a case study using Evo-ROS to optimize placement of sonar sensors on
unmanned ground vehicles that can experience reduced sensing capability due
to component failures and physical damage.
%
The case study provides insights into the current capabilities and identifies areas 
for future enhancements.  
\end{abstract}