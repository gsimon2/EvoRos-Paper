%\vspace{-0.1in}
\section{Background and Related Work}
\label{s:background}
 


% Evolving resilience
% Because it covers a large search space, 
% Evolution is very effective at finding solutions
% beyond preconcptions and therefore 
% bongard, etc.
% -> conclusion?


% sensor placement.
% An importan aspect of resilence is sensing capabiltiy
% Sensor placement is a big issue in robotics

% -> here we exploit that capabilty in placing (sonar)
% sensors on a model of a commercial robot.

\xpkm{PARA 1 (maybe 1\&2): Introduce ER.  Mention/cite works relate to resiliency.  Needs to be 
edited/augmented, as well as smoothed out.}
% Evolutionary robotics (ER)~\cite{Floreano2008} applies principles of biological evolution to the design of robotic systems. 
%
% Individual candidate solutions are evaluated with respect to one or more tasks.  
%
% Evolutionary approaches have produced controllers and physical designs for aquatic, terrestrial, and aerial robots~\cite{bongard-lipson,Lipson2000}.  
%
As with natural evolution, results of many ER studies exhibit a tight coupling between
aspects of morphology, such as sensor positioning, and the controller~\cite{Bongard2015}.
Moreover, the resilience of natural organisms has led researchers to 
apply evolutionary search in order to enhance engineered systems.
Bongard~et~al.~\cite{bongard-lipson} demonstrated the potential of evolution
in {\em self-modeling} terrestrial robots, where the system maintains an internal
``mental image'' of itself and can evolve compensatory behaviors to mitigate damage;
their estimation-exploration algorithm has also been applied to aquatic robots~\cite{Rose.SelfModeling.ERLARS.2013}.
% Model organisms have provided inspiration for aquatic robots~\cite{Clark.ALIFE.2012}, where the flexibility of fins improved performance in a robotic fish.  
% \pkm{Probably mention/cite others.  Maybe our swimming paper?}
Cully~et~al.~\cite{Cully2014} improved and extended this general approach to
enable robots to adapt locomotion strategies in real time, based on sensory feedback.
%


% \paragraph{ER Simulation.}
Despite impressive results from the ER community, however, there remains a disconnect with
the mainstream robotics community, which has also seen major advances in recent years.
\xpkm{cite examples?}
As noted above, ER 
%Simulation is essential to both fields, 
%significantly decreasing the time to 
%evaluate candidate solutions and avoiding damage to a physical platform.
%Within the ER community, 
simulations are typically developed in-house and are simple relative to commercial robots.
% built with tools such as the Open Dynamics Engine~(ODE)~\cite{ODE} and 
% configured per experiment with relatively sparse environments.  
%
% Hence, ER tasks are often limited by the scope of the simulation environment and how much time a developer has to code obstacles, sensors, and the platform itself.  
%
% Moreover, models are not necessarily shareable between developers due to a lack of standardization.  
% 
Silva~et~al.~\cite{Silva.ERissues.2016} recently pointed out these shortcomings and suggested 
possible advantages of adopting tools from mainstream robotics for ER simulations.
In particular, ROS and Gazebo
provide tested models of commercially available actuators and sensors,
saving the ER researcher time in constructing target platforms for evolutionary runs.
Additionally, results have been shown to transfer to real robots, helping
to address the ``reality-gap'' often encountered in ER~\cite{Koos2010}.  
The primary drawback of using high-fidelity simulation in an evolutionary algorithm
is the overhead needed for evaluations.
Our view is that this issue can be partially addressed through 
relatively small-scale parallelization and can eventually be marginalized
with continuing advances in processing capability and larger-scale parallelization.

% \paragraph{Mitigating Sensor Failure}
% \pkm{the GECCO reviewers mentioned other sensor placement studies which we can cite here}
We have realized this approach with the Evo-ROS platform. \xpkm{do we cite the github repository?} \xajc{I suggest we add a placeholder footnote that will eventually link to the github repository but currently is just a comment that the link has been removed to preserve anonymity.}
The particular problem we address in the subsequent case study is optimal placement of sonar sensors
while accounting for possible failures. 
Although these issues are of considerable interest to both the ER and mainstream robotics communities~\cite{Balakrishnan1996,Wang2004,Duckworth.2013.SSRR.Backward}, most studies have focused on fault tolerance and not on sensor placement~\cite{Zhang.2008.FaultTolerance}.
\xpkm{These are pretty old references.  Must be some newer ones from the robotics world.}
% Tony/Jared, can you look for references in this area? See more comments below.}
Evolutionary algorithms are particularly well suited 
to such problems, as they can search large solution spaces, unbiased
by human preconception.
Moreover, evolution can discover ``unlikely-but-possible'' situations that 
might otherwise result in system failure.
\xajc{We should give an example of unlikely-but-possible.}

\xpkm{Any additional transition needed?  Preview results?  Or just leave it and move on to the
details of evoros.}

\xpkm{Should we also comment on any work applying evolution to sensor placement.  
A google search of (robot sensor placement evolution) gives results like these, which I have
not read yet: 
A co-evolution approach to sensor placement and control design for robot obstacle avoidance\\
Evolution of Sensor Placement in Simple Virtual Robots\\
Evolution of Engineering and Information Systems and Their Applications\\
but I suspect placement in the presence of failures is relative open in the ER community, 
though probably better covered in mainstream robotics.}

\xjmm{I added one mainstream and one ER reference in the last paragraph.  Many of the articles I found were on related, but not related enough areas like deploying external sensors in the environment, not on the robot itself.  Perhaps I'm looking under the wrong terms?}

% \pkm{Part 1. Basic flow might follow the ROSCON abstract, to wit:\\
% Evolutionary robotics ->\\
% disconnect with real robots ->\\
% brief intro to Evo-ROS (details to follow in next section->\\
% challenges in terms of speed ->\\
% need for parallelization, which we have done ->//
% (bidirectional) advantages of this approach in applying evolution to real robots
% and making evolution available to roboticists.
% }
% 
% \pkm{Part 2. I think we did a little digging on this and did not find much in the ER
% community, but perhaps you guys know of some relevant works.}
